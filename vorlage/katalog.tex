%%%%%%%%%%%%%%%%%%%%%%%%%%%%%%%%%%%%%%%%%%
% Diese Vorlage steht unter der GPL-Lizenz, Version2 
% siehe  http://www.gnu.de/gpl-ger.html
%%%%%%%%%%%%%%%%%%%%%%%%%%%%%%%%%%%%%%%%%%
\documentclass[a4paper,foldmarks=true]{scrartcl}
\usepackage{fancyhdr}
\usepackage[pdftex]{graphicx}
\usepackage{german}
\usepackage{icomma}
\usepackage{longtable}
\usepackage[utf8]{inputenc}
\usepackage{pifont}
\usepackage{setspace}
\usepackage{textcomp}
\usepackage{ifthen}
\usepackage{filecontents}
\usepackage{ltxtable}
\usepackage{booktabs}
\usepackage{etex}
\usepackage{lastpage}
\usepackage{multirow} 
\usepackage{color}
\setlength{\voffset}{-0.8cm} 
\setlength{\hoffset}{-1.5cm}
\setlength{\topmargin}{0cm}
\setlength{\headheight}{0.5cm}
\setlength{\headsep}{0.1cm}
\setlength{\topskip}{0pt}
\setlength{\oddsidemargin}{1.0cm}
\setlength{\evensidemargin}{1.0cm}
\setlength{\textwidth}{18.1cm}
\setlength{\textheight}{24.3cm}
\setlength{\footskip}{1.8cm} 
\setlength{\parindent}{0pt}
\renewcommand{\baselinestretch}{1}

% ------------------------------------------------------------
% Der eigtl. Dokumenten-Inhalt beginnt hier:
% ------------------------------------------------------------
% Font für Code 39
\font\xlix=wlc39 scaled 1200
\newcommand{\barcode}[1]{{\xlix@#1@}}
\newcommand\WG[1]{%
    \\ \pagebreak
    & \multicolumn{2}{c}{\underline{\textbf{#1}}} & \\ 
}

\newcommand\Zeile[5]{%
     %\multirow{2}{25mm}{ \begin{figure}[ht] \includegraphics*[height=35px]{#1} \end{figure} }& 
     \multirow{2}{25mm}{ \begin{picture}(25,40)(0,-12) \includegraphics*[height=40px]{#1} \end{picture} }& 
     %\multirow{2}{25mm}{ \includegraphics*[height=35px]{#1} }& 
           \multicolumn{2}{p{115mm}}{\small #4} & \small #5 \\
     \nopagebreak & \small #2 & \multicolumn{1}{r}{\small #3 €}   &
            \ifthenelse{\equal{#5}{}}{}{\barcode{#5}}\\
     \\ 
}

\begin{document}
\clearpage
\thispagestyle{empty}
% Schrift-Stil festlegen:
\fontfamily{cmss}\fontsize{12}{15pt plus 0.15pt minus 0.1pt}\selectfont

\centering
\LARGE Katalog \\
%\normal
\vspace{3cm}
\parbox{14cm}{
Meine Firma \\
Überholspur 1\\
\\
12345 Sonstwo\\
\\
Tel.: +49 1234 567890\\
info@meine-firma.false\\
}
\\
\vspace{1cm}
Erstelldatum: <%datum%>

\pagebreak
%Inhaltsverzeichnis
\thispagestyle{empty}
\tableofcontents

% ------------------------------------------------------------
% Die Positionen-Tabelle
% ------------------------------------------------------------
\begin{filecontents}{tabelle.tex}

\begin{longtable}{@{}p{25mm}p{52mm}p{53mm}p{45mm}@{}} % Definition der Spalten

	% Spaltenueberschriften
	%\hline % Trennlinie
    \endfirsthead

	%\hline % Trennlinie
    \endhead

    % Fuss der Teiltabellen
    %\multicolumn{3}{r}{Fortsetzung n\"achste Seite. Seite: \thepage{} von \pageref{LastPage}} \\
  \endfoot

    % Das Ende der Tabelle
    % \hline % Trennlinie
  \endlastfoot

	% Artikel-Positionen:
	<%foreach_part%>   
            \ifthenelse{\equal{<%newpg%>}{new}}{ 
                  \WG{<%partsgroup%>} 
                  \addcontentsline{toc}{section}{<%partsgroup%>} %Eintrags ins Inhaltsverzeichnis
		}
            { \Zeile{<%image%>}{<%partnumber%>}{<%sellprice%>}{<%description%>}{<%ean%>} }
	<%end_part%>

	\hline
\end{longtable}
\end{filecontents}
\LTXtable{\textwidth}{tabelle.tex}


\end{document}
